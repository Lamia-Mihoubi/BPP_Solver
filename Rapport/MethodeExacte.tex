\documentclass[12pt,a4paper, titlepage]{report}
\usepackage[utf8]{inputenc}
\usepackage{graphicx}
\usepackage[french]{babel}
\usepackage[T1]{fontenc}
\usepackage{mathtools}
\usepackage[shortlabels]{enumitem}
\setcounter{tocdepth}{2}
\setcounter{secnumdepth}{3}
% make sure to keep these two lines the very last in the preamble
% if you want to add packages add them before not after these two lines
\usepackage[subpreambles=true]{standalone}
\usepackage{import}
\graphicspath{ {figures/} }
\title{\emph{Rapport du TP OPTIMISATION Pt.1: Methodes Exactes} }
\author{ 
    \emph{BACHI Yasmine (CdE)}
    \and
    SAADI Fatma Zohra Khaoula
    \and 
    NOUALI Sarah
    \and 
    MOUSSAOUI Meroua
    \and 
    MIHOUBI Lamia Zohra}
\begin{document}
    \maketitle
    \tableofcontents
    \chapter*{Introduction}
    \addcontentsline{toc}{chapter}{Introduction} \markboth{INTRODUCTION}{} 
    Le problème du bin packing, dans lequel un ensemble de N objets de différents poids doivent être rangés dans un nombre minimum de boîtes identiques, de capacité C, est un problème NP-difficile, c’est à dire qu’il n y a aucune chance de trouver une méthode de résolution qui fournit la solution exacte en un temps polynomiale, sauf si l’égalité NP=P est prouvée. 
    Durant le dernier siècle, divers efforts ont été consacrés pour étudier ce problème, dans le but de trouver des algorithmes heuristiques rapides pour fournir de bonnes solutions approximatives.
    Dans ce projet, nous allons mettre en place un plateforme de résolution du problème du Bin Packing. pour cela , nous allons implémenter quatre types de méthodes: 
    \begin{enumerate}
        \item \emph{Méthodes exactes:} fournissant la solution optimale, mais qui sont très limitées par la taille du problème. 
        \item \emph{Heuristiques:} qui sont des méthodes approchées spécifiques au problème.
        \item \emph{Métaheuristiques:} qui sont des méthodes approchées génériques.
        \item \emph{Hybridation d'une métaheuristique avec une recherche locale:} qui est notre contribution principale dans la résolution de ce problème.
    \end{enumerate}
    Nous commencerons par la présentation du problème, sa formulation mathématique, et une étude des méthodes de résolutions existantes dans la littérature.[ Etat de l’Art]. Ensuite, nous présenterons la conception détaillée de chaque méthode implémentée, ainsi que les résultats des tests de ces méthodes effectuées sur des benchmarks connus.[Conception \& Tests]
    On distingue deux types de tests:
    \begin{enumerate}
        \item \emph{Les tests empiriques:} dont le but est de trouver la meilleure configuration des paramètres de nos méthodes implémentées.  
        \item \emph{Les tests comparatifs:} où on doit comparer les résultats obtenus des méthodes implémentées et sélectionner la meilleure méthode de résolution pour chaque instance. La comparaison se fait en terme de qualité de la solution et du temps d'exécution. 
    \end{enumerate}
    \chapter{Présentation du Problème de Bin Packing (BPP)}
    \import{chapters/}{Description_BPP}
    \chapter{Etat de l'Art}
    \import{chapters/}{Etat_de_lart}
    \chapter{Conception}
    \import{chapters/}{conception}
    \chapter{Test et Résultats}
    \import{chapters/}{Tests}
    \chapter*{References}
    \addcontentsline{toc}{chapter}{References} \markboth{REFERENCES}{} 
    \begin{enumerate}
        \item Martello ,Toth, \emph{Livre, chapitre 8} \\http://www.or.deis.unibo.it/kp/Chapter8.pdf
        \item Maxence Delorme, Manuel Iori, Silvano Martello, DEI Guglielmo Marconi, \emph{Bin Packing and Cutting Stock Problems: Mathematical Models and Exact Algorithms}, University of Bologna (2)DISMI, University of Modena and Reggio Emilia.
        \item Zoran Ivkovic, Errol L. Lloyd, \emph{FULLY DYNAMIC ALGORITHMS FOR BIN PACKING: BEING (MOSTLY) MYOPIC HELPS}, 1998 Society for Industrial and Applied Mathematics.
        \item E. G. Coman, Jr. M. R. Garey, D. S. Johnson, \emph{APPROXIMATION ALGORITHMS FOR BIN PACKING: A SURVEY}.
        \item Farida Mannai, Mongi Boulehmi, \emph{A Guided Tabu Search for the Vector Bin Packing Problem}, InterVPNC laboratory, FSJEG,University of Jendouba.  
        \item Mirjalili S, Lewis A (2016) \emph{The whale optimization algorithm}.
        \item Mohamed Abdel-Basset, Gunasekaran Manogaran, Laila Abdel-Fatah, Seyedali Mirjalili, (2018) \emph{An improved nature inspired meta-heuristic algorithm for 1-D bin packing problems}, 7 March 2018
        \item R Yesodha , AmudhaT, \emph{Bio-inspired Metaheuristics for Bin Packing Problems}, International Journal of Emerging Technologies in Computational and Applied Sciences (IJETCAS)
    \end{enumerate}
    
\end{document}