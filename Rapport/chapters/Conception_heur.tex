\documentclass[class=article, crop=false]{standalone}
\usepackage{mathtools}
\usepackage{amsmath}
\DeclarePairedDelimiter\ceil{\lceil}{\rceil}
\usepackage[ruled]{algorithm2e}

\usepackage{algorithmic}
\setcounter{section}{0}
\begin{document}
\section{Introduction}
Les heuristiques sont des méthodes spécifiques au problème, qui fournissent des solutions approchées, qui s'imposent car les algorithmes de résolution exactes sont de complexité exponentielle,
et échouent à trouver la solution pour des instances de tailles moyennes voir petites comme on la constater lors du chapitre précédant .
L'usage des heuristiques est donc pertinent pour surmonter ces limites, ces dernières sont utilisées pour calculer une solution approchée du problème, 
mais leur force réelle apparaît lorsqu’on les utilise pour initialiser d’autres méthodes plus génériques. ”.

Dans ce chapitre, nous allons présenter la conception détaillée des heuristiques sur lesquelles notre choix d’implémentation s’est porté et qui sont:
\begin{enumerate}
    \item Next Fit (NF)
    \item Next Fit Decreasing (NFD)
    \item First Fit (FF)
    \item First Fit Descreasing (FFD)
    \item Best Fit (BF)
    \item Best Fit Decreasing (BFD)
\end{enumerate}

Dans le but d’explorer ces méthodes, comparer leurs performances, montrer leurs avantages et découvrir leurs limites , nous effectuerons des tests empiriques et comparatifs sur les mêmes  benchmarks utilisés pour les tests des méthodes exactes.
\newpage
\section{Next Fit (NF)}
\subsection{principe}
Si l'article tient dans la même boite que l'article précédent, il est placé avec ce dernier. Sinon, on ouvre une nouvelle boite et le mettre là-dedans.
\begin{itemize}
    \item NF est un algorithme simple d’une complexité de O(n). 
    \item NF utilise au maximum 2M boîtes, en posant M le nombre de boîtes optimal. 
\end{itemize}

\subsection{Pseudo-Code}
\begin{algorithm}[H]
    \caption{Next Fit}
    \begin{algorithmic}
    \FOR{Tous les articles i = 1, 2,. . . , n}
        \IF{l'article i s'inscrit dans la boîte actuelle}
            \STATE Ranger l’article i dans la boîte actuelle
        \ELSE 
            \STATE Créer une nouvelle boîte, en faire la boîte actuelle et ranger l'article i dedant.
        \ENDIF
    \ENDFOR
    \end{algorithmic}
\end{algorithm}
\section{Next Fit Decreasing (NFD)}
\subsection{principe}
Le NFD est une amélioration de l’algorithme Next-Fit. Cet algorithme ordonne les poids dans le sens décroissant puis applique l’algorithme NF.

\subsection{Pseudo-Code}
\begin{algorithm}
    \caption{Next Fit Decreasing }
    \begin{algorithmic}
        \STATE Triez les objets par ordre décroissant
        \STATE Appliquer Next-Fit à la liste triée d'objets
    \end{algorithmic}
\end{algorithm}

\section{First Fit (FF)}
\subsection{principe}
Ranger chaque article courant dans la première boîte, entre celles déjà ouvertes, qui peut le contenir sinon ouvrir une nouvelle boîte et on le range dedans.
\begin{itemize}
    \item FF utilise au plus 1.7M boîtes, M étant le nombre de boîtes optimal. 
    \item FF est meilleur que NF en termes de limite supérieure sur le nombre de boîtes, ceci est due au nombre de boîtes ouvertes simultanément qui est plus large dans First Fit.
\end{itemize}

\subsection{Pseudo-Code}
\begin{algorithm}[H]
    \caption{First Fit}
    \begin{algorithmic}
    \FOR{Tous les articles i = 1, 2,. . . , n}
        \FOR{Tous les boîtes j = 1, 2,. . . m}
            \IF{l'article i s'inscrit dans la boîte j}
              \STATE Ranger l’article i dans la boîte j
              \STATE Quitter la boucle ( passer à l'article suivant)
             \ENDIF 
        \ENDFOR
        \IF{l’article i ne rentre dans aucune boîte disponible}
            \STATE Créer une nouvelle boîte et ranger l’article i dedans
        \ENDIF
    \ENDFOR
    \end{algorithmic}
\end{algorithm}

\section{First Fit Descreasing (FFD)}
\subsection{principe}
Le FFD est une amélioration de l’algorithme First-Fit . Cet algorithme ordonne les poids dans le sens décroissant puis lui applique l’algorithme FF.
\begin{itemize}
    \item L’algorithme First Fit peut être implémenté en O(nlog n ) en utilisant les arbres de recherche binaires 
\end{itemize}

\subsection{Pseudo-Code}
\begin{algorithm}
    \caption{First Fit Decreasing }
    \begin{algorithmic}
        \STATE Triez les objets par ordre décroissant
        \STATE Appliquer First-Fit à la liste triée d'objets
    \end{algorithmic}
\end{algorithm}


\section{Best Fit (BF)}
\subsection{principe}
Ranger chaque article courant dans la boîte la mieux remplie, entre celles déjà ouvertes, qui peut le contenir sinon ouvrir une nouvelle boîte et on le range dedans.
\begin{itemize}
    \item L’algorithme Best Fit peut être implémenté en O(nlog n ) en utilisant les arbres de recherche binaires
    \item BF utilise au plus 1.7M boîtes, M étant le nombre de boîtes optimal. 
    \item BF est équivalent à l’algorithme FF et meilleur que NF en termes de limite supérieure sur le nombre de boîtes.
\end{itemize}

\subsection{Pseudo-Code}
\begin{algorithm}[H]
    \caption{Best Fit}
    \begin{algorithmic}
    \FOR{Tous les articles i = 1, 2,. . . , n}
        \FOR{Tous les boîtes j = 1, 2,. . . m}
            \IF{l'article i s'inscrit dans la boîte j}
              \STATE Calculer la capacité restante dans la boîte j une fois l'article
             \ENDIF 
        \ENDFOR
        \STATE Ranger l’article i dans la boîte j, où j est la boîte ayant la capacité restante minimale après avoir ajouté l’article(c'est-à-dire que "l’article convient le mieux").
        \IF{une telle boîte n'existe pas ( l’article ne peut être rangé dans aucune boîte)}
            \STATE Créer une nouvelle boîte et ranger l’article i dedans
        \ENDIF
    \ENDFOR
    \end{algorithmic}
\end{algorithm}

\section{Best Fit Decreasing (BFD)}
\subsection{principe}
Le BFD est une amélioration de l’algorithme Best-Fit . Cet algorithme ordonne les poids dans le sens décroissant puis lui applique l’algorithme BF.

\subsection{Pseudo-Code}

\begin{algorithm}
    \caption{Best Fit Decreasing }
    \begin{algorithmic}
        \STATE Triez les objets par ordre décroissant
        \STATE Appliquer Best-Fit à la liste triée d'objets
    \end{algorithmic}
\end{algorithm}

\section{Résultats des tests}
\end{document}