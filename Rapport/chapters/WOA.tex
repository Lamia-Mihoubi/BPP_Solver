\documentclass[12pt]{article}
\usepackage{mathtools}
\usepackage{amsmath}
\usepackage{float}
\usepackage{algorithmic}
\usepackage{algorithm}
\usepackage{hyperref}
\hypersetup{
    colorlinks=true,
    linkcolor=blue,
    filecolor=magenta,      
    urlcolor=cyan,
    bookmarks=true,
}
\setcounter{section}{0}
\setcounter{secnumdepth}{4}
\begin{document}
Le Whale Optimization Algorithm est une nouvelle métaheuristique introduite en 2016 par Mirjalili et Lewis basée sur l’intelligence en essaim. Cet algorithme est inspiré d’une stratégie d’alimentation des baleines à bosse connue sous le nom de \emph{L'alimentation au filet à bulles} . Une tactique qui leur permet d’attraper le plus de poissons possibles en un seul coup. Après avoir détecté ses proies, la baleine libère des bulles en nageant dans un mouvement spirale vers la surface pour encercler la proie pour la capturer.
Les bulles libérées peuvent prendre 2 formes : une forme de cercles rétrécissants ou une forme spirale \textbf{\emph{figure \refeq{fig:alimentation}}}. 
Dans cet algorithme, La recherche des proies représente l’exploration de l’espace de recherche et La libération des bulles représente l’exploitation \cite{mirjalili2016whale}.
\begin{figure}[h!] 
    \includegraphics[width=7cm]{../figures/WOA/cercles.png}
    \includegraphics[width=7cm]{../figures/WOA/spiral.png}
    \caption{Libération des bulles en cercles rétrecissants (à gauche), ibération des bulles en spirale (à droite).}
    \label{fig:alimentation}
\end{figure}
\subsection{Représentation mathématique }
\subsubsection{Encerclement avec bulles en cercles rétrécissants }
Soient:
\begin{itemize}
    \item \(x^*(t)\) la meilleure solution courante.
    \item \(t\) le numéro de l'itération courante.
    \item \(\vec{D}\) indique la distance du ième baleine (ième solution candidate) à la proie (la meilleure solution actuelle).
    \item \(x(t)\) une solution de la population à l'itération \(t\), et \(x(t+1)\) le résultat de sa mise à jour.
    \item \(x_r(t)\) une solution choisie aléatoirement de la population courante.
    \item \(A\) et \(C\) des coefficients calculés par les formules suivantes:
    \[A = 2ar-a\] 
    \[C = 2r\] 
    Avec \(r\) un nombre aléatoire appartenant à \([0,1]\).
    \\ et \(a\) un nombre décrémenté linéairement à chaque itération de 2 à 0, de la façon suivante:
    \(a =  a_{init} - a_{init}*i/max\_iter\) avec \(i\) numéro de l’itération courante, \(max\_iter\) le nombre maximal des itération et \(a_{init}\) la valeur initiale de \(a\).
\end{itemize}
Le processus d’encerclement de la proie peut être représenté par les équations suivantes :

\begin{equation*}
    \vec{D} = |Cx^*(t) - x(t)|
\end{equation*}
Si \(|A| < 1\):
\begin{equation}
    x(t+1) = x^*(t) - A\vec{D}
\end{equation}
Sinon:
\begin{equation}
    x(t+1) = x_r(t) - A\vec{D}
\end{equation}
Le comportement de l’encerclement avec bulles en cercle rétrécissant est obtenu en diminuant la valeur de \(a\) de \(a_{init}\) à 0 au cours des itérations. La variation de \(A\) peut être utilisée pour rechercher des proies, c'est-à-dire la phase d'exploration. Par conséquent, \(A\) peut être utilisée avec des valeurs aléatoires supérieures à 1 ou inférieures à -1 pour forcer les solutions à s'éloigner de la solutions de référence (la meilleure solution \(|A| < 1\) ou une solution aléatoire sinon).
\subsubsection{Encerclement avec bulles sous forme spirale }
Il est modélisé par les équations suivantes :
\begin{equation}
    \vec{D} = |x^*(t) - x(t)|
\end{equation}
\begin{equation}
    x(t+1) = \vec{D} e^{bl} \cos (2\pi l) + x^*(t)
\end{equation}
Où \(l\) est un nombre aléatoire appartenant à \([-1,1]\) Et \(b\) une constante qui définit la forme de la spirale 

Ces deux types définissent deux mécanismes d’exploration de l’espace de recherche, ce qui permet une meilleure diversification de l’espace de recherche.
Dans chaque itération du WOA, un de ces deux mécanismes est choisi avec une probabilité \(p\) égale à \(50\%\) pour mettre à jour la population des solutions candidates comme suit:
\[ x(t+1) =
  \begin{cases}
    x^*(t) - A\vec{D}       & \quad r < p\\
    \vec{D} e^{bl} \cos (2\pi l) + x^*(t)  & \quad r \geq p
  \end{cases}
\]
avec \(x*(t)\) est la meilleure solution actuelle au temps \(t\), \(p\) est égale à \(0,5\) et \(r\) est un nombre aléatoire entre \([0, 1]\)

\subsection{Ingrédients du WOA }
\begin{itemize}
    \item Population initiale des solutions.
    \item Fonction d’évaluation qui permet de choisir la meilleure solution x*(t).
    \item Un mécanisme d'évolution:
    \begin{itemize}
        \item Encerclement avec bulles en cercle rétrécissant.
        \item Encerclement avec bulles sous forme de spirale.
    \end{itemize}
    \item Critère d'arrêt du WOA: nombre maximal d’itérations.    
\end{itemize}
\subsection{Paramètres du WOA }
\begin{itemize}
    \item La taille de la population des solutions candidates à évaluer dans chaque itération.
    \item La constante \(b\) qui définit la forme de la spirale.
    \item Le nombre maximal d’itérations: \(max\_iter\).
    \item Le nombre \(a\) qui détermine le degré de diversification (exploration).    
\end{itemize}
\subsection{Application de l’algorithme au problème du bin packing :}
\subsubsection{Discrétisation de l'espace de recherche}
Cet algorithme a été proposé pour la résolution de problèmes à espace \\ de recherche continu. Afin de l’adapter à notre problème discret nous allons utiliser une méthode appelée LOV qui permet de passer d’une solution continue à une solution discrète:\\
\begin{algorithm}[H]
    \caption{Discrétisation de l'espace de recherche par LOV}
    \begin{algorithmic}
        \STATE \textbf{Résultat en sortie:} Solution discrète \(X = (x_1, x_2, \dots, x_n)\) obtenue à partir de la solution continue \(\tilde{X} = (\tilde{x_1}, \tilde{x_2}, \dots, \tilde{x_n})\)\;
        \STATE Ordonner les valeurs du vecteur de \(\tilde{X}\) par ordre décroissant\;
        \STATE L'indice d'ordre de chaque valeur est stocké dans un vecteur \(\theta = \{ \theta_i = \text{ordre de l'élément } \tilde{x_i}\}\)\;
    \end{algorithmic}  
\end{algorithm}
\subsubsection{Représentation d’une solution }
Une solution est exprimée par un vecteur $ x(t)=(a_1, a_2, …, a_n)$ représentant la distribution des articles dans les boîtes au moment t, avec n le nombre d’articles et $a_i$ est un article d’ordre i. Les articles sont ordonnés selon les boîtes dans lesquelles ils sont rangés, i.e. l’article $a_1$ est rangé dans la première boîte, s’il y’en a de l’espace dans cette boîte alors l’article $a_2$ est y rangé, sinon $a_2$ est rangé dan la deuxième boîte...ainsi de suite. 
Une solution \(x(t)\) appartient au domaine de recherche tant qu’elle respecte les contraintes du problème: un objet ne peut pas être rangé dans plus d’une boîte, donc pas de doublons dans x(t), l’autre contrainte concernant le respect de la capacité d’une boîte est vérifiée trivialement par définition du vecteur x(t).

\subsubsection{La fonction objective }
Afin de pouvoir évaluer les solutions, Nous avons opté pour la fonction suivante proposée par \emph{Hyde et al} \cite{hyde2009hyflex}, au lieu du nombre de boîtes utilisées, parce que avec cette dernière pour plusieurs solutions on peut avoir la même évaluation ce qui peut engendrer la stagnation de l’algorithme.
\[ F_{min} = 1 - \frac{\sum_{1}^{n} (occup_i / c)^k}{n}\]
Avec:
\begin{itemize}
    \item \(n\) nombre de boites utilisées.
    \item \(ocup_i\) total des poids des objets rangés dans l’ième.
    \item \(c\) capacité des boites.
    
\end{itemize}
\subsection{Pseudocode}
\begin{algorithm}[H]
    \caption{Whale Optimization Algorithm}
    \begin{algorithmic}
        \STATE Initialiser la population de baleines aléatoirement (l'ensemble initial des solutions candidates): Xi (i = 1, 2 ,..., N)\;
        \STATE Évaluer les solutions de la population initiale\;
        \STATE X* = la meilleure solution actuelle\;
        \WHILE{\(t < max\_iter\)}
            \FOR{solution \(\in\) population}
                \STATE Mettre à jour a, A, C, l et r\;
                \IF{\(r  < 0,5\)}
                    \IF{\(|A| < 1\)}
                        \STATE Mettre à jour la solution par Eq.(1)\;
                    \ELSE
                        \STATE Sélectionnez une solution aléatoire \(X_r\)\;
                        \STATE Mettre à jour la solution par Eq.(2)\;
                    \ENDIF
                \ELSE 
                    \STATE Mettre à jour la solution par l'Eq.(3)\;
                \ENDIF
            \ENDFOR
            \STATE Vérifier si une solution dans la population dépasse l'espace de recherche et la modifier\;
            \STATE Appliquer la discretisation par LOV\;
            \STATE Évaluer la nouvelle population\;
            \STATE Mettre à jour X* \;
            \STATE t = t + 1\;
        \ENDWHILE
    \end{algorithmic}
\end{algorithm}
\subsection{Test et résultats }
Dans cette partie nous allons tester les performances de la métaheuristique WOA sur les instances du benchmark Scholl. Ensuite nous allons effectuer une comparaison des résultats obtenus avec les résultats optimaux. Nous utiliserons pour chacune des trois classes du benchmark Scholl deux configurations de paramètres; la première configuration est obtenue par tâtonnement après plusieurs essais manuels, la deuxième par calibrage automatiques des paramètres utilisant le package \emph{irace} qui implémente l’algorithme I/F-Race (voir section \textbf{Calibrage Automatique des Paramètres}).
Pour pouvoir étudier les performances de WOA, notre étude se comportera 2 axes:
\begin{itemize}
    \item Temps d'exécution.
    \item Qualité de la solution  (ratio et comparaisons avec les solutions optimales)
\end{itemize}

\subsubsection{Rappel des paramètres de WOA }
\begin{itemize}
    \item La taille de la population: \(nb\_whales\).
    \item Le nombre maximal d’itérations: \(max\_iter\).
    \item La constante de l’encerclement spirale \(b\).
    \item La constante \(a\) qui détermine le degré d’exploration de l’espace de recherche.
\end{itemize}
\subsubsection{Temps d'exécution }
\begin{figure}[h!]
    \includegraphics[width=10cm]{../figures/WOA/woa_texec.png}
    \caption{Histogramme des temps d'exécutions moyens (en s) pour les deux configurations pour tout le benchmark Scholl. Configuration 1 en bleu. Configuration 2 en rouge.}
    \label{fig:texec}
\end{figure}
Configurations de paramètres utilisées dans le graphe \textbf{\emph{figure \refeq{fig:texec}}}:
\begin{enumerate}
    \item Configuration 1 (obtenue manuellement): \(nb\_whales=30, max\_iter=50, b=1.5, a=4\).
    \item Configuration 2 (obtenue par calibrage automatique par IRACE): \( nb\_whales=30, max\_iter=117, b=8.96, a=10\).
\end{enumerate}

\subsubsection{Qualité de solution }
\paragraph{Ratio: }
\begin{figure}[h!]
    \includegraphics[width=10cm]{../figures/WOA/ratio_woa_configs.png}
    \caption{Histogramme des ratios des deux configurations pour tout le benchmark Scholl. Configuration 1 en bleu. Configuration 2 en rouge.}
    \label{fig:ratio}
\end{figure}
Configurations de paramètres utilisées dans le graphe \textbf{\emph{figure \refeq{fig:ratio}}}:
\begin{enumerate}
    \item Configuration 1 (obtenue manuellement): \(nb\_whales=30, max\_iter=50, b=1.5, a=4\).
    \item Configuration 2 (obtenue par calibrage automatique par IRACE): \(nb\_whales=28, max\_iter=271,  b=7.64, a=20\).
\end{enumerate}

\paragraph{Comparaison avec la solution optimale: }
\begin{figure}[h!]
    \includegraphics[width=14cm]{../figures/WOA/woa_opt.png}
    \caption{le nombre de boites obtenu par WOA (en vert) par rapport à la solution optimale (en bleu) pour les instances de la classe 1 et 2 de taille respectivement de haut en bas, de gauche à droite: N1, N2, N3, N4 et les instances de la classe 3.}
    \label{fig:compare}
\end{figure}
Configuration de paramètres utilisée dans les graphes \textbf{\emph{figure \refeq{fig:compare}}} (obtenue par calibrage automatique par IRACE) : \(nb\_whales=28, max\_iter=271,  b=7.64, a=20\).

\subsubsection{Analyses des résultats }
\begin{itemize}
    \item WOA permet d'obtenir un ratio inférieur à 1.4 pour les benchmark Scholl, et un ratio de 1 pour les instances de la classe 2 de taille N1.
    \item Parmis les trois classes WOA est le plus performant dans le cas des instances difficiles (classe 3: C3) et la classe 2.
    \item Le calibrage automatique permet généralement d’améliorer la qualité de la solution très légèrement sauf pour le cas des instances de la classe C2 de tailles N1 et N4 ou la configuration manuelle donne de meilleurs résultats.
    \item Le calibrage automatique permet de réduire le temps d’exécution significativement surtout pour les instances de grande taille (N4) et les instances de la classe 3 (instances difficiles).
    \item Avec une configuration optimale des paramètres le temps de résolution des instances les plus difficiles et/ou volumineuses ne dépasse pas 8s.
\end{itemize}
\end{document}