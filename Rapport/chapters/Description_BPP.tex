\documentclass[12pt]{article}
\renewcommand{\thesection}{\arabic{section}}
\usepackage{mathtools}
% make sure to keep these two lines the very last in the preamble
% if you want to add packages add them before not after these two lines
\usepackage[subpreambles=true]{standalone}
\usepackage{import}
\begin{document}
    \section{Domaines d'Application}
Le BPP a de nombreuses applications dans le domaine industriel, informatique, etc. Parmi lesquelles on cite:
    \renewcommand{\labelitemi}{$\circ$}  
    \begin{itemize}
        \item Chargement de conteneurs.
        \item Placement des données sur plusieurs disques.
        \item Planification des travaux.
        \item Emballage de publicités dans des stations de radio / télévision de longueur fixe.
        \item Stockage d’une grande collection de musique sur des cassettes / CD, etc.
    \end{itemize}
    \section{Formulation Mathématique}
    Etant donné \(m\) boites de capacité \(C\) et \(n\) articles de volume \(v_i\) chacun. \\
    Soient: 
    \[ x_{ij} =
        \begin{cases}
            1  & \quad \text{article } j \text{ rangé dans la boîte } i \\
            0  & \quad \text{sinon } 
        \end{cases}
    \]
\[ y_i =
    \begin{cases}
        1  & \quad \text{boîte } i \text{ utilisée } \\
        0  & \quad \text{sinon } 
    \end{cases}
\]

La formulation du problème donne ainsi le programme linéaire suivant
\[(PN)
    \begin{cases}
        Z(min) = \displaystyle\sum_{i=1}^{m} y_i \\
        \displaystyle\sum_{i=1}^{m} x_{ij}  = 1 \\
        \displaystyle\sum_{j=1}^{n} v_j x_{ij} \le C y_i \\
        y_i \in \{0,1\} \\
        x_{ij} \in \{0,1\} 
    \end{cases}
\]  
La première contrainte signifie qu’un article j ne peut être placé qu’en une seule boite
La deuxième fait qu’on ne dépasse pas la taille d’une boite lors du rangement

\end{document}