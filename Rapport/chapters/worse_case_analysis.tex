\documentclass[class=article, crop=false]{standalone}
\usepackage{mathtools}
\usepackage{amsmath}
\usepackage{import}
\usepackage{float}
\setcounter{section}{0}
\begin{document}

Les méthodes approchées cherchent à trouver une solution la plus proche possible de la solution optimale,
pour mesurer la qualité de cette solution obtenue, nous utiliserons l’analyse du pire des cas \emph{“worst Case analysis”}.

Dans cette analyse, les performances d’un algorithme sont mesurées par l’écart de la solution du pire cas
(l’instance où l’algorithme donne la pire solution) à la solution optimale.
L’une des métriques les plus utilisées dans l’analyse du pire des cas est le \emph{Worst Case Ratio}.

\section*{Worst Case Ratio}
Ce rapport mesure la déviation maximal de la solution obtenue par l’heuristique, par rapport à la solution optimale. 

Soit :
\begin{itemize}
    \item \textbf{L}: une instance du bin packing.
    \item \textbf{A(L)}: le nombre de boîtes utilisées en appliquant l’heuristique \emph{A} sur \emph{L}.
    \item \textbf{OPT(L)}: le nombre de boîtes optimal.
\end{itemize}
Le rapport est donné par la formule suivante :\\
\[ Ra\equiv \{r \geq 1 : \frac{A(L)}{OPT(L)} \geq r \; pour\: toute\: instance\: L\}\]\\
Pour calculer le ratio, on calcul le rapport  $A(L)/OPT(L)$  pour chaque instance, ensuite on prend le plus petit des majorants de ces rapports.
Il est claire que la valeur idéale est un “1”, ce qui correspond au cas où la solution obtenue est  égale à la solution optimale pour toutes les instances, et dès que la solution obtenue s’éloigne de la solution optimale le rapport va augmenter.
\end{document}